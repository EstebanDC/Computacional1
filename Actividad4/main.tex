\documentclass{article}
\usepackage[spanish]{babel}
\usepackage[utf8]{inputenc}
\usepackage[T1]{fontenc}
\usepackage{graphicx}
\graphicspath{{images/}}
\usepackage{float}
\title{\begin{figure}[h]
\includegraphics[width=8cm]{UNI}
\centering
\end{figure}
UNIVERSIDAD DE SONORA\\
\vspace{0.2cm}
LICENCIATURA EN FÍSICA\\
\vspace{0.2cm}
FÍSICA COMPUTACIONAL\\
\vspace{0.7cm}
\huge{Python y tefigramas}}\vspace{0.2cm}
\author{Prof. Carlos Lizárraga Celaya\\\vspace{0.17cm}
Esteban Delgado Curiel}

\begin{document}
\maketitle
\newpage
\begin{centering}
\section*{Resumen}
\end{centering}
El intentar interpretar el comportamiento de la naturaleza es un gran reto, y son necesarias un gran número de herramientas para intentar comprender la pequeña información que ésta nos brinda. Una de las formas de interpretar los datos obtenidos por los distintos sondeos realizados a la atmósfera es por medio de gráficas. 
\title{\begin{figure}[H]
\includegraphics[width=14cm]{Tephi}
\centering
\end{figure}
\begin{centering}
\vspace{1cm}
\section*{Introducción}
En esta actividad se llevará a cabo el uso de tefigramas, que son uno de los cuatro tipos  de diagramas termodinámicos comunmente usados para el análisis del clima. Ésto para interpretar los datos obtenidos de sondeos realizados en la ciudad de Guadalajra, Jalisco, y se llevará a cabo la programación en Python utilizando Jupyter Notebook.
\vspace{1cm}
\end{centering}
\vspace{1cm}
\section{Procedimiento para la obtención de datos}\vspace{0.5cm}
\subsection{Guadalajara, Jal. Observations}
El tefigrama que se relizará será en base a la interpretación de los datos obtenidos los sondeos atmosféricos realizados en Guadalajara, Jalisco, el 20 de febrero del 2017. Los cuales se mostrarán en la siguiente tabla:\\
\title{\begin{figure}[H]
\includegraphics[width=13cm]{Datos}
\centering
\end{figure}

\newpage
\section{Gráficas}
La creación de las gráficas se llevó a cabo en Jupyter Notebook, utilizando los comandos presentados a continuación:

\subsection{Presión vs Altura}
\title{\begin{figure}[H]
\includegraphics[width=12cm]{PvsA}
\centering
\caption{Presión vs Altura}
\end{figure}
Comandos: 
\begin{verbatim}
Asignación de los ejes correspondientes a la gráfica:

x=df[u'Altura']
y=df[u'Presión']

Elaboración de la gráfica:

mplt.plot(x,y)
mplt.grid(True)
plt.xlabel('Altura (m)')
plt.ylabel('Presión (hPa)')

Mostrar la gráfica:

plt.show()
\end{verbatim}}

\newpage
\subsection{Temperatura vs Altura}
\title{\begin{figure}[H]
\includegraphics[width=13cm]{TvsA}
\centering
\caption{Temperatura vs Altura}
\end{figure}
Comandos: 
\begin{verbatim}
Asignación de los ejes correspondientes a la gráfica:

y=df[u'Altura']
x=df[u'Temperatura']

Elaboración de la gráfica:

mplt.plot(x,y)
mplt.grid(True)
plt.xlabel('Temperatura (C)')
plt.ylabel('Altura (m)')


Mostrar la gráfica:

plt.show()
\end{verbatim}

\newpage
\subsection{DWPT vs Altura}
\title{\begin{figure}[H]
\includegraphics[width=13cm]{DWPTvsA}
\centering
\caption{DWPT vs Altura}
\end{figure}
Comandos: 
\begin{verbatim}
Asignación de los ejes correspondientes a la gráfica:

y=df[u'Altura']
x=df[u'DWPT']

Elaboración de la gráfica:

mplt.plot(x,y)
mplt.grid(True)
plt.ylabel('Altitud')
plt.xlabel('DWPT (C)')


Mostrar la gráfica:

plt.show()
\end{verbatim}

\newpage
\subsection{DWPT-Temperatura vs Altura}
\title{\begin{figure}[H]
\includegraphics[width=13cm]{2TvsA}
\centering
\caption{DWPT-Temperatura vs Altura}
\end{figure}
Comandos: 
\begin{verbatim}
Elaboración de la gráfica:

y=df[u'Altura']
x=df[u'Temperatura']
mplt.plot(x,y)
mplt.grid(True)
plt.xlabel('Temperatura (C)')
plt.ylabel('Altura (m)')

y=df[u'Altura']
x=df[u'DWPT']
mplt.plot(x,y)
mplt.grid(True)
plt.ylabel('Altitud')
plt.xlabel('DWPT (C)')

Mostrar la gráfica:

plt.show()
\end{verbatim}

\section{Tefigrama}
\title{\begin{figure}[H]
\includegraphics[width=13cm]{Tephi}
\centering
\caption{Presión vs DWPT- Presión vs Temperatura}
\end{figure}
Comandos: 
\begin{verbatim}
Importar Tephi:

import os.path
import tephi as tph

Elaboración del tefigrama:

dew_point = pd.read_csv("/home/estebandelgado/Fisica_Computacional/
Actividad4/PvsDWPT.csv", names=["Presión", "DWPT"])
dry_bulb = pd.read_csv("/home/estebandelgado/Fisica_Computacional/
Actividad4/PvsT.csv", names=["Presión", "TEMP"])
tpg = tph.Tephigram()
tpg.plot(dew_point)
tpg.plot(dry_bulb)

Mostrar tefigrama:

plt.show()
\end{verbatim}

El siguiente comando fue utilizado para la instalación de Tephi desde la terminal:
\begin{verbatim}
pip install --upgrade pip --user /home/estebandelgado/
Fisica_Computacional/Actividad4/tephi
\end{verbatim}

\section{Conclusión}
 El uso Jupyter Notebook puede ser de gran ayuda en cuanto a la interpretación de los datos y de la mano de los tefigramas dan una mejor visualización de los datos obtenidos por los sondeos realizados, facilitando el manejo de los mismos.

\newpage
\begin{thebibliography}{9}
\bibitem{m1}
 \textit{http://weather.uwyo.edu/upperair/sounding.html}
 \bibitem{m2}
\textit{https://en.wikipedia.org/wiki/Tephigram}
\end{thebibliography}

\end{document}