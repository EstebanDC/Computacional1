\documentclass{article}
\usepackage[spanish]{babel}
\usepackage[utf8]{inputenc}
\usepackage[T1]{fontenc}
\usepackage{graphicx}
\graphicspath{{images/}}
\usepackage{float}
\title{\begin{figure}[h]
\includegraphics[width=8cm]{UNI}
\centering
\end{figure}
UNIVERSIDAD DE SONORA\\
\vspace{0.2cm}
LICENCIATURA EN FÍSICA\\
\vspace{0.2cm}
FÍSICA COMPUTACIONAL\\
\vspace{0.7cm}
\huge{Nuestra Atmósfera}}\vspace{0.2cm}
\author{Prof. Carlos Lizárraga Celaya\\\vspace{0.17cm}
Esteban Delgado Curiel}

\begin{document}
\maketitle
\newpage
\begin{centering}
\section*{Resumen}
\end{centering}
La atmósfera es una pieza fundamental en el mecanismo que se encarga de que se lleve a cabo la vida en nuestro planeta. En el transcurso de estas páginas se intentará dar a conocer al lector la importancia que tiene el conocer a fondo la composición y estructura de nuestra atmósfera, al igual que su papel en nuestro día a día y el impacto que recibe debido las consecuencias que provocamos los seres humanos.
La atmósfera ayuda a que la vida sea posible en este planeta, y tiene mucho más que ofrecer, solo debemos conocerla más a fondo y hacer todo lo posible por mantenerla en funcionabilidad. 
\vspace{1.5cm}

\begin{centering}
\section*{Introducción}
\end{centering}
En la actualidad se han presentado un gran número de problemas en cuanto
a la estabilidad del clima en nuestro planeta, entre ellos el calentamiento global,
el efecto invernadero, y muchos otros que poco a poco aumentan su nivel de
gravedad. Todos y cada uno de ellos provocados en gran parte por la insistencia
del ser humano en alterar la naturaleza a su conveniencia, pasando por alto
que algún día las consecuencias podrían ser irreversibles, y que con el paso del
tiempo aumentarán los daños que van directamente dirigidos hacia lo más cercano
que tenemos como protección para la Tierra, nuestra atmósfera.
\vspace{1.5cm}

\section{Concepto y estructura}\vspace{0.5cm}
\subsection{¿Qué es la atmósfera?}
Llamamos atmósfera a una mezcla de varios gases que rodea cualquier obje-
to celeste, como la Tierra, cuando éste posee una campo gravitatorio suficiente
para impedir que escapen. En la Tierra, la actual mezcla de gases se ha desa-
rrollado a lo largo de 4500 millones de años. A lo largo de este tiempo, diversos
procesos físicos, químicos y biológicos transformaron esa atmósfera primitiva
hasta dejarla tal como ahora la conocemos. Además de proteger el planeta y
proporcionar los gases que necesitan los seres vivos, la atmósfera determina el estado del tiempo y el clima.\\


\subsection{Estructura de la atmósfera}
La estructura de la atmósfera condiciona la forma en la que ésta se comporta, además, controla el desarrollo del clima cerca de la superficie de la Tierra. \\
Las moléculas que cubren la atmósfera son atraídas a la superficie de la Tierra por medio de la fuerza gavitatoria. Esto ocasiona que la atmósfera se concentre en la superficie de la Tierra y se haga más delgada conforme aumente la altura.\\
\\
Tomando como criterio la composición química de la atmósfera, ésta puede ser dividida en dos capas, homosfera y heterosfera. La homosfera ocupa desde la superficie de la Tierra hasta una altura de 100 km, y cuenta con una composición química muy uniforme. La heterosfera se extiende desde los 100 km hasta el límite superior de la atmósfera (en algunas zonas unos 10,000 km); además está estratificada, es decir, formada por diversas capas de composición diferente.  

\title{\begin{figure}[H]
\includegraphics[width=10cm]{structure}
\centering
\end{figure}
La atmósfera se compone de 4 capas: La troposfera, estratosfera, mesosfera y termosfera.\\

\begin{itemize}
\item \large{Troposfera:}\vspace{0.2cm}\\
Es la capa más baja de la atmósfera, en esta capa es donde vivimos y donde se lleva a cabo el proceso para generar el clima. La temperatura en esta capa generalmente decrece con la altura. El límite entre la estratosfera y la toposfera es llamada la tropopausa. La altura de la troposfera varía con la locación, aumentando sobre áreas cálidas y disminuyendo sobre áreas con temperaturas bajas. 
\item \large{Estratosfera:}\vspace{0.2cm}\\
Sobre la toposfera se encuentra la estratosfera. En esta capa la temperatura incrementa con la altura. Esto debido a que la estratosfera alberga la preciada capa de ozono. La capa de ozono tiene una alta temperatura ya que absorbe los rayos ultravioleta provenientes del sol. 
\item \large{Mesosfera:}\vspace{0.2cm}\\
La mesosfera es la capa que se encuentra sobre la estratosfera. En esta capa, de manera igual a la troposfera y contraria a la estratosfera, la temperatura decrece con la altura. La diferencia es que contiene un poco de vapor de agua, por lo que el aire es muy ligero y no se presta para llevar a cabo el proceso que genera el clima.   
\item \large{Termosfera:}\vspace{0.2cm}\\
La termosfera es la capa más elevada de la atmósfera. En esta capa la temperatura incrementa con la altura ya que esta siendo impactada de forma directa por el sol.
\end{itemize}
\vspace{1cm}

\section{\Large{La Capa de ozono}}
La capa de ozono es la encargada de protegernos de los rayos ultravioleta provenientes del sol, que son los principales causantes del cáncer de piel entre otros problemas de salud.\\
Ésta capa es definida por los meteorólogos como una región donde se concentra el ozono, y que se encuentra justo sobre la troposfera.

\title{\begin{figure}[H]
\includegraphics[width=10cm]{Ozono}
\centering
\end{figure}
Aproximadamente el 90\% del ozono de la atmósfera se encuentra en la estratosfera. El ozono puede ser un arma de dos filos, cuando esta en la superficie de la Tierra puede llegar a ser tanto peligroso como corrosivo al momento de respirarlo. Pero cuando éste se concentra en la troposfera ayuda a protegernos de los dañinos rayos UV. \\
Sin ozono en la atmósfera, sería muy peligroso salir a caminar sin tener que usar algún tipo de protección.

\section{Gases y efecto invernadero}
El efecto invernadero es producido por gases atmosféricos que absorben energía tanto del sol como de la tierra y la retiene cerca de la superficie terrestre, siendo pieza clave de una labor sumamente crucial para nuestra atmósfera, mantener al planeta lo suficientemente caliente para que éste pueda albergar vida. El efecto invernadero no es lo mismo que el calentamiento global, ya que al hablar sobre el calentamiento global nos referimos al incremento en la temperatura promedio en el planeta, debido a enormes cantidades de gases de invernadero.\\
Los gases de invernadero más abundantes que son responsables del efecto invernadero en la atmósfera son el vapor de agua, dióxido de carbono, metano, óxido nitroso, y ozono. Estos gases de invernadero mantienen la superficie de la Tierra aproximadamente 15.5 grados celsius más caliente de lo que se espera sin ellos.

\subsection{¿Cuál es el proceso del efecto invernadero?}
Primeramente, la energía del sol pasa por la atmósfera como pequeñas ondas de radiación hasta llegar al suelo. Después el suelo, las nubes y otras superficies terrestres absorben esta energía y la liberan de regreso hacia el espacio en forma de grandes ondas de radiación. Conforme estas grandes ondas se elevan en la atmósfera, son absorbidas por los gases de invernadero, hasta que eventualmente abandonan la atmósfera. Desde que parte de la radiación es emitida de vuelta a la superficie de la Tierra, la radiación se calienta más de lo que se calentaría si no estuvieran los gases de invernadero presentes.\\
\begin{enumerate}
\item 
\end{enumerate}

Si los gases de invernadero aumentan su concentración y no cambia algo en la atmósfera, muy probablemente la temperatura de la superficie tienda a aumentar. La cantidad de radiación dirigida de regreso a la Tierra aumentaría y el balance de energía se vería alterado. Por lo que, sabiendo que la Tierra tiene un sistema climatológico muy complicado, si sucede este incremento de energía, otros aspectos como el incremento de evaporación y la formación de nubes, así como el derretimiento de los polos, serán más propensos a ocurrir, de formas en las que cambiaría tanto regional como globalmente el clima y las temperaturas.

\section{La Naturaleza}
En mi opinión, la solución simplemente simula estar oculta, pero se encuentra justo de frente a nuestros ojos, en nuestros alrededores, en los bosques, plantas, fenónemos naturales, en alguna parte del conjunto de elementos que forman nuestro hogar, es decir, en la naturaleza, y se mantiene ahí esperando a ser encontrada por alguno de nosotros, y lo que debemos hacer es poner todo nuestro esfuerzo en intentar comprenderla un poco más, para que nos muestre sus secretos y podamos aprovecharlos en beneficio de todos, y de paso revertir los daños que le hemos ocasionado. 

\newpage

\section*{Bibliografía}
\begin{itemize}
\item \large{www.astromia.com - atmosfera}\vspace{0.2cm}
\item \large{Structure of the Atmosphere. NC State University.}\vspace{0.2cm}
\item \large{e-ducativa.catedu.es - Composición de la atmósfera}\vspace{0.2cm}
\item \large{Ozone Layer. NC State University}\vspace{0.2cm}
\item \large{Greenhouse Gases. NC State Univeristy}\vspace{0.2cm}
\item \large{Greenhouse Effect. NC State University}\vspace{0.2cm}

\end{itemize}
\end{document}
