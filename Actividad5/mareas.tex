\documentclass[12pt]{article}
\usepackage[spanish]{babel}
\usepackage[utf8x]{inputenc}
\usepackage{graphicx} 
\usepackage{graphicx}
\usepackage{caption}
\usepackage{subcaption}
\usepackage{float}

\title{Mareas y Corrientes}		% Título
\author{\centering Delgado Curiel Esteban}											% Autores
\date{28 de Marzo del 2017} %Aquí pueden cambiarla%					% Fecha de edición

\makeatletter
\let\thetitle\@title
\let\theauthor\@author
\let\thedate\@date										
\makeatother

%% Para que salga el título en todas las hojas
%\pagestyle{fancy}
%\fancyhf{}
%\lhead{\thetitle}
%\cfoot{\thepage}

\begin{document}

%%%%%%%%%%%%%%%%%%%%%%%%%%%%%%%%%%%%%%%%%%%%%%%%%%%%%%%%%%%%%%%%%%%%%%%%%%%%%%%%%%%%%%%%%

\begin{titlepage}
	\begin{centering}
    
    \vspace*{0.5 cm}
    \includegraphics[scale = 0.5]{logouni}\\[0.5 cm]	% University Logo
    \textsc{\Large Universidad de Sonora}\\[1.0 cm]	% University Name
	\textsc{\Large División de Ciencias Exactas y Naturales}\\[0.5 cm]				% Course Code
	\textsc{\large Física Computacional}\\[0.5 cm]				% Course Name
	\rule{\linewidth}{0.2 mm} \\[0.4 cm]
	{ \huge \bfseries \thetitle}\\
	\rule{\linewidth}{0.2 mm} \\[0.5 cm]
	
	\begin{minipage}{\textwidth}
		\begin{flushleft} 
			\emph{\Large} \large \\
			\theauthor
			\end{flushleft}
	
		\begin{flushleft} 
			{Profesor:} \large \centering Carlos Lizárraga Celaya	
			\end{flushleft}
	\end{minipage}\\[1 cm]
	{\large \thedate}\\[2 cm]
 
	\vfill
    \end{centering}
    \end{titlepage}
    
\section*{Breve resumen}
Dejando atrás los sondeos atmósfericos ahora se estudiará el fenómeno de la variación de los niveles del mar en varios puntos costeros para estudiar la dinámica de las mareas dada por las fuerzas gravitacionales de la Luna, el Sol, la rotación de la Tierra y otras fuerzas externas. Para el estudio de las mareas se utilizarán los datos de las ciudades Coatzacoalcos, Veracruz, y Boston, Massachusets. Datos que se obtendrán de la página del CICESE (Centro de Investigación Científica y de Educación Superior de Ensenada) y del NOAA Tides \& Currents (National Ocenic and Atmospheric Administration - Administración Nacional Oceánica y Atmosférica).

\section*{Introducción}
Uno de los fenómenos mas estudiados y conocidos está en las masas de agua que forma parte de aproximadamente el 70\% del planeta tierra teniendose en cuenta que estos pueden movilizarse y cambiar el comportamiento tanto por factores internos del planeta, por factores externos, relativos a la posición del planeta respecto a otros o la posición de su satélite natural. \\

\vspace{1cm}

\begin{figure}[h]
\centering
\includegraphics[width=15cm]{Eins}
\end{figure}

 \newpage
 
 \section{Mareas}
 \vspace{0.35cm}
 \subsection{Definición}
La marea se define como la 
oscilación periódica del nivel 
del mar que resulta de la 
atracción gravitacional de la 
Luna y el Sol que actúa sobre 
la Tierra en rotación.
\vspace{0.25cm}
 
 \subsubsection{Etapas del proceso de las mareas}
 
 Son 4 etapas del proceso de las mareas:
\begin{enumerate}
\item El nivel del mar se eleva paulatinamente durante varias horas.

\item El nivel del agua alcanza su nivel más alto.

\item El nivel del agua desciende poco a poco durante varias horas.

\item El agua deja de descender y alcanza su nivel más bajo.
\end{enumerate}

Las mareas no se presentan únicamente en los océanos, pueden ser provocadas en otros sistemas de agua siempre y cuando se den las condiciones necesarias.

\section{Componente lunar semidiurno principal}
La a Marea Semidiurna se caracteriza por dos mareas altas y dos mareas bajas durante el día lunar, por eso el periodo 
es igual a la mitad del día lunar (12 h 25 m). Las alturas de las mareas altas y bajas siguientes 
tienen muy poca diferencia, debido a que la marea semidiurna se superpone a la marea diurna.

Este componente representa la rotación de la Tierra respecto a la Luna. Velocidad = 28,984 104 2° por hora solar.

\section{Variación de las mareas: marea muerta y de primavera (neap and spring tides)}

Las mareas de primavera se dan dos veces por cada mes lunar en el transcurso del año sin considerar la estación del año. Las mareas muertas, las cuales se dan dos veces al mes, son provocadas cuando el Sol y la Luna se encuentran a $90\,^{\circ}$ uno del otro.\\

\begin{figure}[H]
  \begin{subfigure}[b]{0.5\textwidth}
\includegraphics[width=\textwidth]{Spring}
    \caption{Marea de primavera.}
    \label{fig:f1}
  \end{subfigure}
  \hfill
  \begin{subfigure}[b]{0.6\textwidth}
    \includegraphics[width=\textwidth]{Neap}
    \caption{Marea muerta.}
    \label{fig:f2}
  \end{subfigure}
\end{figure}

\section{Altitud lunar}
%The changing distance separating the Moon and Earth also affects tide heights. When the Moon is closest, at perigee, the range increases, and when it is at apogee, the range shrinks. Every  7 1⁄2 lunations (the full cycles from full moon to new to full), perigee coincides with either a new or full moon causing perigean spring tides with the largest tidal range.
\begin{figure}[h]
\centering
\includegraphics[width=6cm]{luna}
\end{figure}

La separación entre la Luna y la Tierra también afecta la altura de las mareas.  Cuando la Luna se
encuentra lo más cercana posible a la Tierra, en el perigeo, el rango incrementa, y cuando esta alejada, el rango disminuye. 


%\section{Fase y Amplitud}
%Because the M2 tidal constituent dominates in most locations, the stage or phase of a tide, denoted by the time in hours after high water, is a useful concept. Tidal stage is also measured in degrees, with 360° per tidal cycle. Lines of constant tidal phase are called cotidal lines, which are analogous to contour lines of constant altitude on topographical maps. High water is reached simultaneously along the cotidal lines extending from the coast out into the ocean, and cotidal lines (and hence tidal phases) advance along the coast. Semi-diurnal and long phase constituents are measured from high water, diurnal from maximum flood tide. This and the discussion that follows is precisely true only for a single tidal constituent.
%Debido a que el componente lunar semidiurno principal conocido por el simbolo $M_2$ domina en la mayoría de los lugares, la fase de la marea, denotada por el tiempo en horas después de agua elevada, es un conpeto bastante útil. Las fases de las mareas también son medidas en grados, siendo $360\,^{\circ}$ por cada ciclo de la marea. Las lineas constantes de marea son llamadas
 
 

\section{Física en las mareas}
Isaac Newton (1642-1727) fue la primera persona en explicar las mareas como producto de la atracción gravitacional entre masas asotronómicas. Su explicación de las mareas y otros fenómenos fue publicada en el Principia en 1687 y utilizó su teoría de gravitación universal para explicar a la atracción de la luna y el sol como el origen de la fuerza en las mareas. 

Otro dato interesante es que en 1740, Académie Royale des Sciences de París, Francia, ofreció un premio para el mejor ensayo teórico sobre mareas. Daniel Bernoulli, Leonhard Euler, Colin Maclaurin and Antoine Cavalleri compartieron el galardón.

\begin{figure}[h]
\centering
\includegraphics[width=9cm]{Field}
\end{figure}

\section{Fuerzas}
La fuerza de las mareas producidas por un objeto con masa (Luna) sobre una pequeña partícula ubicada dentro o sobre un cuerpo (Tierra) es el vector que hace la diferencia entre la fuerza gravitacional de la Luna ejercida sobre una partícula, y la fuerza gravitacional que sería ejercida sobre la partícula si ésta estuviera ubicada en centro de masa de la Tierra. La fuerza gravitacional solar sobre la Tierra es en promedio 179 veces más fuerte que la lunar, pero debido a que el Sol esta aproximadamente 389 veces más alejado de la Tierra, su campo gradiente es más débil. La fuerza en las mareas provocada por el sol es equivalente a un $\%46$ de las provocadas por la Luna. 


\section{Ecuaciones de Laplace sobre las mareas}
En 1776, Pierre-Simon Laplace formuló un sistema de ecuaciones lineales parciales para describir el flujo de las mareas. El efecto Coriollis esta introducido al igual de la fuerza gravitacional. 

%For a fluid sheet of average thickness D, the vertical tidal elevation ζ, as well as the horizontal velocity components u and v (in the latitude φ and longitude λ directions, respectively) satisfy Laplace's tidal equations:[28]
Para un fluido de espesor $D$, elevación vertical de la marea $\xi$, así como las componentes horizontales de la velocidad $u$ y $v$ (en la latitud $\varphi$ y longitud $\lambda$, respectivamente) se satisfacen las ecuaciones de Laplace sobre las mareas:


	\begin{equation}
	\frac{\partial					\xi}{\partial{t}}+\frac{1}		{a\cos{\varphi}}				\left[\frac{\partial}			{\partial \lambda}				(uD)+\frac{\partial}			{\partial	 \varphi}			(vD\cos{\phi}) \right]=0,
	\end{equation}
    
    \begin{equation}
	\frac{\partial					u}{\partial{t}}-v(2\Omega		\sin{\varphi})+\frac{1}			{a\cos{\varphi}}				\frac{\partial}					{\partial \lambda}(g\xi+U)=0
	\end{equation}
    
    \begin{equation}
	\frac{\partial					v}{\partial{t}}+u(2\Omega		\sin{\varphi})+\frac{1}			{a}\frac{\partial}				{\partial \varphi}(g\xi+U)=0
	\end{equation}
    \vspace{1cm}

Donde $\Omega$ es la frencuencia angular de la rotación del planeta, $g$ es la aceleración gravitacional del planeta en la superficie del oceano, $a$ es el radio planetario, y $U$ es el potencial gravitacional externo de la fuerza de las mareas.


    \begin{figure}[H]
  \begin{subfigure}[b]{0.4\textwidth}
\includegraphics[width=\textwidth]{moon1}
    \caption{Vista sobre el hemisferio Norte del globo terrestre.}
    \label{fig:f1}
  \end{subfigure}
  \hfill
  \begin{subfigure}[b]{0.4\textwidth}
    \includegraphics[width=\textwidth]{moon2}
    \caption{Globo terrestre rotado $180^{\circ}$.}
    \label{dddd}
  \end{subfigure}
  \caption{Potencial gravitacional lunar}
\end{figure}


\newpage
\textbf{\section{\Large Principales Componentes Armónicos de las Mareas}}
% TABLA 
\begin{table}[htbp]
\begin{center}
\begin{tabular}{|l|l|l|}
\hline \hline
Nombre & Símbolo & Periodo (hrs)  \\
\hline \hline
Límite de agua superficial de la luna principal & $M_{4}$ & 6.210300601 \\ \hline
Límite de agua superficial de la luna principal & $M_{6}$ & 4.140200401 	\\ \hline
Agua superficial terdiurnal & $MK_{3}$ & 8.177140247 \\ \hline
Abundancia de agua poco profunda de la energía solar principal & $S_{3}$ & 6 \\ \hline
Cuarto de agua poco profunda diurna &  $MN_{4}$  & 6.269173724 \\ \hline
Principal lunar semidiurno & $M_{2}$ & 12.4206012 \\ \hline
Principal solar semidiurno & $S_{2}$ & 12 \\ \hline
Gran lunar elíptica semidiurna & $N_{2}$ & 12.65834751 \\ \hline
Lunar diurno & $K_{1}$ & 23.93447213  \\ \hline
Lunar diurno & $O_{1}$ & 25.81933871 \\ \hline
\end{tabular}
\label{tabla:sencilla}
\end{center}
\end{table}

\begin{thebibliography}{9}

\bibitem{a1} \textsc{https://www.importancia.org/mareas.php} \textit{Mareas}

\bibitem{a1} \textsc{http://www.oceanografia-gral-fis.at.fcen.uba.ar/TP4-Biol\&Geol/Mareas\_2010.pdf} \textit{Definción de mareas}

\textbf{\bibitem{a1} \textsc{https://en.wikipedia.org/wiki/Tide} \textit{Mareas}
}

\textbf{\bibitem{a1} \textsc{https://en.wikipedia.org/wiki/Theory$\_$of$\_$tides} \textit{Mareas}
}

\end{thebibliography}

\end{document}