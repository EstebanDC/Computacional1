\documentclass[12pt]{article}
\usepackage[spanish]{babel}
\usepackage[utf8x]{inputenc}

\title{Preguntas Actividad 2}
\author{Dr. Carlos Lizárraga Celaya\\ Esteban Delgado Curiel}
\date{08 de Febrero del 2017}
\begin{document}
\maketitle

1.-¿Cuál es tu primera impresión del uso de bash/Emacs?\\
Parecer un poco complejo pero conforme se va trabajo con él se reconoce su facilidad.\\

2.-¿Ya lo habías utilizado? \\
Sí, en el curso de programación y lenguaje Fortran.\\

3.-¿Qué cosas se te dificultaron más en bash/Emacs?  \\
Comandos que nuevos y algunas herramientas que aún no aprendo a usar bien.\\

4.-¿Qué ventajas les ves a Emacs?\\
Es muy útil y práctico. Además de ser muy fácil de usar, es muy efectivo.\\

5.-¿Qué es lo que mas te llamó la atención en el desarrollo de esta actividad?\\
El número de utilidades que se le puede dar a Emacs, y la forma de trabajar con datos reales.\\

6.-¿Qué cambiarías en esta actividad?\\
Nada.\\

7.-¿Que consideras que falta en esta actividad?\\ 
Practicar de manera más pausada las distintas utilidades que ofrece Emacs.\\
    
8.-¿Puedes compartir alguna referencia nueva que consideras útil y no se haya contemplado?\\ 
No, si mal no recuerdo utilizo fuentes ya contempladas. \\

9.-¿Algún comentario adicional que desees compartir? \\
Me gustaría que en las prácticas siguientes se mantengan los temas relacionados a la atmósfera, al iguaaprender a reconocer ciertas características sobre los lugares asignados por medio de la observación de los datos.
\end{document}
