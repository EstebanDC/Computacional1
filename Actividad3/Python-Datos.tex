\documentclass[12pt]{article}
\usepackage[spanish]{babel}
\usepackage{url}
\usepackage[utf8x]{inputenc}
\usepackage{amsmath}
\usepackage{subfig}
\usepackage{graphicx}
\graphicspath{{images/}}
\usepackage[rightcaption]{sidecap}
\usepackage{parskip}
\usepackage{float}
\usepackage{fancyhdr}
\usepackage{vmargin}
\usepackage[usenames]{color}
\setmarginsrb{3 cm}{2.5 cm}{3 cm}{2.5 cm}{1 cm}{1.5 cm}{1 cm}{1.5 cm}

\title{Manejo de datos con Python}	% Título% 
\date{08 de Febrero del 2017} %Aquí pueden cambiarla%					% Fecha de edición
\makeatletter
\let\thetitle\@title
\let\theauthor\@author
\let\thedate\@date										
\makeatother

\pagestyle{fancy}
\fancyhf{}
\lhead{\thetitle}
\cfoot{\thepage}

\begin{document}

%%%%%%%%%%%%%%%%%%%%%%%%%%%%%%%%%%%%%%%%%%%%%%%%%%%%%%%%%%%%%%%%%%%%%%%%%%%%%%%%%%%%%%%%%

\begin{titlepage}
	\centering
    \vspace*{0.5 cm}
    \includegraphics[scale = 0.5]{logo}\\[0.5 cm]	% University Logo
    \textsc{\Large Universidad de Sonora}\\[0.5 cm]	% University Name
	\textsc{\Large División de Ciencias Exactas y Naturales}\\[0.5 cm]				% Course Code
    \textsc{\Large Licenciatura en Física}\\[0.5 cm]	% 
	\textsc{\large Física Computacional}\\[0.5 cm]	% Course Name
	\rule{\linewidth}{0.2 mm} \\[0.4 cm]
	{ \huge \bfseries \thetitle}\\
	\rule{\linewidth}{0.2 mm} \\[0.5 cm]
	
	\begin{minipage}{\textwidth}
		\begin{centering} 
			\emph{\Large  Dr. Carlos Lizárraga Celaya \\
            \vspace{0.5cm} Esteban Delgado Curiel} \\
			\end{centering}
            
	\end{minipage}\\[1 cm]
	{\large \thedate}\\[2 cm]
 
	\vfill
	
\end{titlepage}

%======================================================================================-

%======================================================================================-
\begin{center}
\section*{Breve resumen}
\end{center}
En esta práctica se utilizará la información brindada por los sondeos atmosféricos de la ciudad correspondiente, en este caso Guadalajara, Jalisco, con el fin de organizar e interpretar el significado de la variación que estos presentan, filtrando los resultado obtenidos sobre CAPE y la cantidad de Agua precipitable.

\begin{centering}
\section*{Introducción}
\end{centering}
En esta práctica se lleva a cabo la elaboración de gráficas y diagramas de caja para las variables CAPE y Agua precipitable utilizando Jupyter Notebook, el cual es una aplicación en línea que permite crear y compartir documentos que contienen códigos, ecuaciones, visualizaciones y textos. También permite llevar a cabo la organización y limpieza de datos, simulaciones numéricas, modelos estadísticos, entre otras cosas más.\\

\begin{centering}
\section{Datos}
\includegraphics[width=10cm]{Tabla2.png}

\subsection{}
\begin{centering}
\includegraphics[width=10cm]{Tabla1.png}
\end{centering}
\end{centering}



\newpage
\section{CAPE}
\begin{center}
\includegraphics[width=13.4cm]{CAPE}
\end{center}
CAPE es un indicador de inestabilidad atmosférica que se considera muy valioso en cuanto a predecir condiciones del clima. Y el Agua precipitable es el contenido de humedad en la atmósfera; se mide como el espesor vertical que ocuparía si toda el agua cayera. 
\vspace{1cm}
\begin{figure}[H]
 \centering
  \subfloat[00Z]{
   \label{00Z}
    \includegraphics[width=0.48\textwidth]{CAPEG}}
  \subfloat[12Z]{
   \label{12Z}
    \includegraphics[width=0.48\textwidth]{CAPEG2}}
    \caption{Gráficas CAPE}
\end{figure}
\begin{figure}[H]
 \centering
  \subfloat[00Z]{
   \label{00Z}
    \includegraphics[width=0.48\textwidth]{CAPEG3}}
  \subfloat[12Z]{
   \label{12Z}
    \includegraphics[width=0.48\textwidth]{CAPEG4}}
    \caption{Gráficas CAPE}
\end{figure}

\subsection{Agua precipitable}
\begin{center}
\includegraphics[width=13.4cm]{AP}
\end{center}
El agua precipitable es la cantidad de agua, expresada como altura o masa, que se obtendría si todo el vapor de agua contenido en una columna específica de la atmósfera, de sección transversal horizontal unitaria, se condensase y precipitase. \\

\begin{figure}[H]
 \centering
  \subfloat[00Z]{
   \label{00Z}
    \includegraphics[width=0.58\textwidth]{AG1}}
  \subfloat[12Z]{
   \label{12Z}
    \includegraphics[width=0.58\textwidth]{AG2}}
 \caption{Gráficas Agua precipitable}
 \label{f:animales}
\end{figure}

\section{Procedimiento para la obtención de datos}
Para elaborar la limpieza y organización de datos se utilizó el editor de textos Emacs, el cual es muy práctico y útil más que nada.\\
\\
Para filtrar los datos obtenidos de los sondeos atmosféricos que utilizaron los siguientes comando de Emacs:
\begin{verbatim}

cat sondeos.txt | egrep -i "Observations|CAPE|precipitable" |
sed -e '/00Z/,+2d' > 12Zanual.txt

cat sondeos.txt | egrep -i "Observations|CAPE|precipitable" |
sed -e '/12Z/,+2d' > 00Zanual.txt
\end{verbatim}
\vspace{0.5cm}
Después se utilizaron comandos en Emacs como "query replace" para extraer solo datos útiles de los filtrados anteriormente para después agregarlos a nuevos archivos.

\newpage
Tomando los archivos creados en Emacs, en Python se crearon las gráficas mostradas a lo largo de este reporte utilizando los siguientes comandos:

\begin{verbatim}
import pandas as pd
import numpy as np
import matplotlib as plt
df = pd.read_csv("/home/estebandelgado/Física Computacional/Actividad3
/00Zanual.csv")

df.head(21)
df.describe()
df.apply(lambda x: sum(x.isnull()),axis=0)
df_clean = df.dropna()
df_clean.describe()
df.columns
matplotlib inline
df_clean['CAPE'].hist(bins=21)
df.CAPE.dtypes
df.boxplot(column='CAPE', return_type="axes") 
df.boxplot(column='Agua precipitable',return_type="axes") 
\end{verbatim}
Esto para la obtención de datos en 00Z



\begin{thebibliography}{9}
\bibitem{m2}
 \textit{http://weather.uwyo.edu/upperair/sounding.html}, \textit{Radiosondeos}
\end{thebibliography}

\end{document}